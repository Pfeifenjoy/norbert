%!TEX root = Zusammenfassung.tex

\documentclass[%
	pdftex,
	oneside,			% Einseitiger Druck.
	11pt,				% Schriftgroesse
	parskip=half,		% Halbe Zeile Abstand zwischen Absätzen.
	headsepline,		% Linie nach Kopfzeile.
	footsepline,		% Linie vor Fusszeile.
	abstracton,			% Abstract Überschriften
	listof=totoc,
	toc=bibliography,
	headings=optiontohead,
	plainfootsepline
]{scrreprt}


\usepackage{xstring}
\usepackage[utf8]{inputenc}
\usepackage[T1]{fontenc}
\usepackage{lmodern}
\renewcommand{\familydefault}{\sfdefault}

\usepackage{sansmath}
\sansmath
\usepackage{here}

\usepackage[ngerman]{babel}

\usepackage[a4paper, left=25mm,right=25mm, top=38mm, bottom=43mm]{geometry}
\usepackage[activate]{microtype} %Zeilenumbruch und mehr
\usepackage{setspace}
\usepackage{makeidx}
\usepackage[autostyle=true,german=quotes]{csquotes}
\usepackage{longtable}
\usepackage{graphicx}
\usepackage{xcolor} 	%xcolor für HTML-Notation
%\usepackage{float}
\usepackage{array}
\usepackage[right]{eurosym}
\usepackage{wrapfig}
\usepackage[perpage, hang, multiple, stable]{footmisc}
\usepackage[printonlyused,footnote]{acronym}
\usepackage{listings}
%\lstset{language=C}
\lstset{showstringspaces = false}
\usepackage{epstopdf}
\usepackage{nameref}


\usepackage{textcomp}

\usepackage{rotating}

\usepackage{lscape}

%% Farben (Angabe in HTML-Notation mit großen Buchstaben)
\definecolor{LinkColor}{HTML}{00007A}
\definecolor{ListingBackground}{HTML}{FCF7DE}

%% Mathematikpakete benutzen (Pakete aktivieren)
\usepackage{amsmath}
\usepackage{amsthm}
\usepackage{amsbsy}
\usepackage{amssymb}

%% Programmiersprachen Highlighting (Listings)
\newcommand{\listingsettings}{%
	\lstset{%
		language=Java,			% Standardsprache des Quellcodes
		%numbers=left,			% Zeilennummern links
		stepnumber=1,			% Jede Zeile nummerieren.
		numbersep=5pt,			% 5pt Abstand zum Quellcode
		numberstyle=\tiny,		% Zeichengrösse 'tiny' für die Nummern.
		breaklines=true,		% Zeilen umbrechen wenn notwendig.
		breakautoindent=true,	% Nach dem Zeilenumbruch Zeile einrücken.
		postbreak=\space,		% Bei Leerzeichen umbrechen.
		tabsize=2,				% Tabulatorgrösse 2
		basicstyle=\ttfamily\footnotesize, % Nichtproportionale Schrift, klein für den Quellcode
		showspaces=false,		% Leerzeichen nicht anzeigen.
		showstringspaces=false,	% Leerzeichen auch in Strings ('') nicht anzeigen.
		extendedchars=true,		% Alle Zeichen vom Latin1 Zeichensatz anzeigen.
		captionpos=b,			% sets the caption-position to bottom
		backgroundcolor=\color{ListingBackground}, % Hintergrundfarbe des Quellcodes setzen.
		xleftmargin=0pt,		% Rand links
		xrightmargin=0pt,		% Rand rechts
		frame=single,			% Rahmen an
		frameround=ffff,
		rulecolor=\color{darkgray},	% Rahmenfarbe
		fillcolor=\color{ListingBackground}
	}
}



% PDF Einstellungen
\usepackage[%
	pdftitle={Zusammenfassung Software Engineering I},
	pdfauthor={Philipp Pütz},
	pdfcreator={pdflatex, LaTeX with KOMA-Script},
	pdfpagemode=UseOutlines, 		% Beim Oeffnen Inhaltsverzeichnis anzeigen
	pdfdisplaydoctitle=true, 		% Dokumenttitel statt Dateiname anzeigen.
	pdflang={de}, 			% Sprache des Dokuments.
]{hyperref}

% (Farb-)einstellungen für die Links im PDF
\hypersetup{%
	colorlinks=true, 		% Aktivieren von farbigen Links im Dokument
	linkcolor=LinkColor, 	% Farbe festlegen
	citecolor=LinkColor,
	filecolor=LinkColor,
	menucolor=LinkColor,
	urlcolor=LinkColor,
	linktocpage=true, 		% Nicht der Text sondern die Seitenzahlen in Verzeichnissen klickbar
	bookmarksnumbered=true 	% Überschriftsnummerierung im PDF Inhalt anzeigen.
}
% Workaround um Fehler in Hyperref, muss hier stehen bleiben
\usepackage{bookmark} %nur ein latex-Durchlauf für die Aktualisierung von Verzeichnissen nötig

% Schriftart in Captions etwas kleiner
\addtokomafont{caption}{\small}

%%%%%% Additional settings %%%%%%

% Hurenkinder und Schusterjungen verhindern
% http://projekte.dante.de/DanteFAQ/Silbentrennung
\clubpenalty=10000
\widowpenalty=10000
\displaywidowpenalty=10000


\usepackage[autooneside=false,automark]{scrlayer-scrpage}
   
\clearpairofpagestyles
\cfoot*{\pagemark}
\ihead{\ifstr{\rightbotmark}{\leftmark}{}{\rightbotmark}}
\ohead{\leftmark}

\title{\textbf{Zusammenfassung Software Engineering I}}
\author{Philipp Pütz}
\date{\today}