%!TEX root = ../documentation.tex

\chapter{Qualitätsziele}
Als grundsätzliche Qualitätsziele sollen die im folgenden spezifizierten Punkte erfüllt werden.

\begin{tabularx}{\textwidth}{|l|X|l|}
\toprule
\textbf{ID} & \textbf{Beschreibung}\\
\endhead
\hline
Q-10 & Flüssig steuerbare Anwendung  \\
Q-20 & Intuitive Bedienbarkeit  \\
Q-30 & Robuste Anwendung  \\
Q-40 & Dokumentierte Anwendung  \\  
Q-50 & Toleranz bei Eingabefehlern \\
Q-60 & Unterstützung von verschiedenen Plattformen (Mobile Endgeräte, Desktop-PCs) \\
Q-70 & Unterstützung von verschiedenen Browsern (Firefox, Chrome) \\
Q-80 & Hohe Performance \\
Q-80-1 & Geringe Dauer von Datenbanktransaktionen (< 500ms)\\
Q-90 & Modifizierbarkeit \\
Q-90-1 & Möglichkeit der Anpassung der Software an neue Anforderungen/Funktionen\\
Q-90-2 & Möglichkeit besteht, einzelne Komponenten einfach auszutauschen\\
Q-100 & Sicherheit \\
Q-100-1 & Gesetzliche Datenschutzanforderungen werden eingehalten \\
Q-100-2 & Server- und Softwareinfrastruktur ist vor unberechtigten Zugriff abgesichert \\
Q-110 & Die Software ist Effizient und kann auch auf kleineren Serversystemen betrieben werden \\
\hline
\end{tabularx}