%!TEX root = ../documentation.tex

\chapter{Einleitung}
Norbert - Your StudyBuddy ist eine Softwarelösung zum Verwalten des Studienalltags. Nobert soll auf die speziellen Anforderungen eines Studierenden angepasst sein und ihm bessere Möglichkeiten zum Meistern des Studiums bieten. Dabei wird auf die speziellen Bedürfnisse eines dualen Studierenden an der Dualen Hochschule Baden Württemberg (DHBW) eingegangen.

Jeder kennt das Problem: Während des Studiums gibt es mehrere Plattformen auf denen Informationen zu finden sind. 
Dabei ist es meist schwer den Überblick zu behalten and anfallende Aufgaben zu managen.
Jeder Professor hat dabei seine eigene Methode den Studenten Informationen und Aufgaben zukommen zu lassen. Der Student muss dann diese Informationen filtern und aufbereiten so dass er sich einen Plan für das Studium gestalten kann.
Dieser Prozess soll durch Norbert unterstützt werden. 
Die Software schlägt dabei dem Studenten auf Grund einer Textanalyse Aufgaben vor oder Hilf ihm beim durchsuchen der Informationen von verschiedenen Plattformen (z.B. E-Mail oder Dropbox).
Dadurch soll die Organisation des Studenten vereinfacht werden wodurch er den effektiver sich auf die fachlichen Inhalte des Studiums konzentrieren kann.

%Jeder kennt das Problem: Der Studierende hat sich einen Studiengang ausgesucht und ein duales Partnerunternehmen gefunden. Doch wie geht es weiter? Die Möglichkeit, direkt Zugang zu Vorlesungsinhalten, Informationen zu Dozenten, Informationen über die DHBW oder den Kursplan zu erhalten, besteht nicht. Die Studierenden müssen erst mühsam die Informationen aus unzähligen unübersichtlichen DHBW - Webseiten heraussuchen. Dabei stoßen sie auf eine Vielzahl an PDF-Dokumenten, die sie später einmal benötigen werden. Doch welcher Studierende denkt schon zu Beginn des ersten Semesters an die Abgabe des Praxisberichts I 9 Monate später?

%Genau an diesen Punkten setzt die Softwarelösung Norbert - Your StudyBuddy an. Sie bietet den Studierenden die Möglichkeit ihren Studienalltag zu strukturieren und zu planen. Durch den Austausch von Informationen und Aufgaben mit Studierenden in höheren Semestern gestaltet sich das Zurechtfinden und Eingewöhnen in die neue Lernumgebung einfacherer. Durch die kontinuierliche Verwendung unserer Anwendung und die Nutzung der bereitgestellten Funktionen kann der Studienalltag besser geplant werden und wertvolle Zeit im Bereich der Organisation eingespart werden. Diese zusätzliche Freizeit kann für Hobbys, Kneipentouren oder zum effektiveren Lernen der Vorlesungsinhalte genutzt werden. Letztlich wird auch die Chance erhöht, nicht in den ersten Semestern überfordert zu sein und nach den ersten Klausuren aussteigen zu müssen. 
