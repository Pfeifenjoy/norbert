\chapter{Funktionen}


\begin{table}[H]
    %\caption{ID Kapitel Zuweisung} %TODO
    \label{Funktionen::main}
    \begin{longtable}{|l|l|l|l}
        \toprule
        \textbf{ID-Kürzel} & \textbf{Beschreibung} & \textbf{Prio} & \textbf{Abhängig von} \\
        \endhead
        \hline
        F-00 & Es gibt eine Software & 1 & \\
        F-00.1 & Mehrere Benutzer können auf die Software zugreifen, ohne sie installieren zu müssen. & 1 & \\
        F-00.2 & Die Software kann an Desktopgeräten und mobilen Geräten verwendet werden. & 1 & \\
        \hline
        F-10 & Die Software dient als persönlicher Assistent während des Studiums. & & \\
        F-10 & Planen von Aufgaben & 1 & \\
        F-10.1 & Es können Aufgaben erstellt werden. & 1 & \\

        F-20 & Dem Benutzer werden hilfreiche Informationen über das Studium angezeigt. & 1 & \\

        F-30 & Die Software kann Erinnerungen anzeigen. & 1 & \\

        F-40 & Die Software kann Push-Notifications an den Benutzer senden. & & \\

        F-50 & Es gibt einen Administrationsbereich & & \\
        F-50.1 & Der Administrationsbereich ist durch ein Kennwort geschützt. & & \\ % Kennwort wurde während der Installation festgelegt
        F-50.2 & Es können Benutzer durch die Eingabe von Emailadressen hinterlegt werden. & & \\
        F-50.2.1 & Die so erstellten Benutzer erhalten über eine Email Zugriff auf das System. & & \\ % TODO: Formulierung
        
        F-60 & Die Software kann auf Datenquellen, wie z.B. Google-Drive, DHBW Seite, Moodle etc. zugreifen. & & \\

        F-70 & Die Software kann Informationen zwischen den Studenten austauschen. & & \\
        F-70.1 & Todo's von anderen Studenten werden angezeigt. & &\\
        F-70.2 & Es können Kursnachrichten versendet werden. & &\\
        F-80 & Es können Bezüge zu Dateien aus Dropbox etc. hergestellt werden. & &\\
        F-90 & User werden anhand ihres Namens markiert & &\\ %Noch unklar

        F-* & Formulare & &\\

        F-* & Todo's können privat sein. & & \\

        

        % Todo liste analysieren ob es andere gibt.
        % ToDo: Prozessdiagramm



        \hline
    \end{longtable}
\end{table}

Datums-termine aus Dokumenten finden
PDFs parsen
Dokumente anhand von Keywörtern gruppieren -> Vorschlag für weitere Quellen.
Vorschlag von dokumenten anhand von tasks erstellen.

Google Places anzeigen -> Pizza, Bars, Festivals

