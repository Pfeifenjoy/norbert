\chapter{Funktionen}

\section{Allgemeine Funktionsweise}

Im Folgenden sollen die Funktionsweise und Begrifflichkeiten der Anwendung Norbert erläutert werden.
Norbert dient als \enquote{Study Buddy}, Unterstützer eines DHBW-Studenten.

Um die grundlegende Benutzung zu verstehen, soll zunächst skizziert werden, wie Norbert mit den Benutzer interagiert:

Benutzer können in Norbert Informationen wie zum Beispiel Aufgaben, Erinnerungen oder Notizen ablegen. Diese werden als \enquote{Einträge} bezeichnet. Die Einträge aller Nutzer werden zentral gesammelt.

Aufgrund der vorhandenen Daten prognostiziert Norbert, welche Informationen für welchen Benutzer relevant sein könnten. Dabei kann es sich um Einträge anderer Benutzer handeln, aber auch andere Informationsquellen, wie der Speiseplan der Kantine, werden berücksichtigt.

Jedem Benutzer werden die für ihn relevanten Informationen angezeigt. Handelt es sich dabei um die Einträge anderer Benutzer, schlägt Norbert vor, diese direkt in die eigenen Einträge zu übernehmen. (Vergleiche Abbildung~\ref{funktionen:datenaustausch})

\begin{figure}[H]
    \centering
    \includegraphics[width=0.8\textwidth]{images/funktionsweise.png}
    \caption{Datenaustausch}
    \label{funktionen:datenaustausch}
\end{figure}

Ein Eintrag kann aus mehreren Komponenten bestehen. Je nach Typ der jeweiligen Komponenten können diese unterschiedliche Informationstypen speichern oder Funktionalitäten erfüllen. Tabelle~\ref{funktionen:typen} gibt eine Übersicht über alle Komponententypen.

\begin{tabularx}{\textwidth}{|l|l|X|l|}
    \toprule
    \textbf{ID-Kürzel} & \textbf{Name} & \textbf{Beschreibung} & \textbf{Prio}\\
    \midrule
    \endhead
    \hline
    \caption{Typen von Komponenten}
    \label{funktionen:typen}
    \endfoot
    F-Comp-00 & Text & Kann zur Beschreibung des Eintrags verwendet werden. & 1\\
    F-Comp-10 & Erinnerung & Ermöglicht, einen Zeitpunkt zu definieren, an dem der Benutzer an den Eintrag erinnert wird. & 2 \\
    F-Comp-20 & Aufgabe & Ermöglicht es, den Eintrag als \enquote{erledigt} oder \enquote{nicht erledigt} zu markieren. & 1\\
    F-Comp-30 & Ort & Dem Eintrag wird ein Ort zugewiesen. & 2 \\
    F-Comp-40 & Bild & Es wird ein Bild in einem Eintrag angezeigt. & 2 \\
    F-Comp-50 & Dokument & Es kann ein Dokument an den Eintrag angehängt werden. & 1\\
    F-Comp-60 & Link & Der Eintrag verweist auf eine Internetadresse. & 2\\
    %F-Types-30 & Teilen & Der Eintrag kann mit anderen Benutzern geteilt werden.
    %Wenn jemand anderes diesen Eintrag editiert, wird dieser auch bei dem Ersteller des Eintrags editiert. & 1\\
\end{tabularx}

%Das Konzept eines solchen Eintrags soll anhand der Abbildung \ref{funktionen:Eintrag} verdeutlicht werden. 
%Hier ist zu sehen das ein Eintrag modular aufgebaut ist. Standardmäßig muss für einen Eintrag ein Titel vergeben werden.
%Die in Tabelle \ref{funktionen:typen} genannten Typen können durch hinzufügen von Feldern realisiert werden. Zum Beispiel wird eine Erinnerung durch hinzufügen eines Datums oder eines Ortes erstellt.

%\begin{figure}[H]
%    \centering
%    \includegraphics[width=0.4\textwidth]{images/beispiel-eintrag.png}
%    \caption{Funktionsweise-Eintrag}
%    \label{funktionen:Eintrag}
%
%\end{figure}

Der \enquote{Newsfeed} ist der Sammelpunkt, an dem jedem Benutzer alle Informationen angezeigt werden, die für ihn relevant sind. Er enthält:

\begin{enumerate}
    \item Die Einträge des Benutzers.
    \item Vorschläge, welche Einträge anderer Benutzer in die eigenen Einträge übernommen werden sollten.
    \item Informationen aus externen Informationsquellen.
\end{enumerate}

Die Informationen werden innerhalb des Feeds nach Relevanz sortiert angezeigt. Die Relevanz kann dabei von der Wichtigkeit oder der Aktualität der Informationen abhängen. Der Newsfeed wird von Norbert automatisch aktuell gehalten.

\newpage
\section{Spezifische Funktionen}
Im Folgenden werden spezifische Funktionen aufgelistet und beschrieben:

\begin{tabularx}{\textwidth}{|l|X|l|l|}
    \toprule
    \textbf{ID-Kürzel} & \textbf{Beschreibung} & \textbf{Prio} & \textbf{Abhängig von} \\
    \midrule
    \endhead
    \hline
    \caption{Funktionen}
    \endfoot
    \multicolumn{4}{|l|}{F-00 Allgemein}\\
    \hline
    F-00.1 & Das Produkt ist eine Softwarelösung. & 1 & \\
    F-00.2 & Mehrere Benutzer können auf die Software zugreifen. & 1 & \\
    F-00.3 & Die Daten werden auf einem zentralen Server gespeichert. & 1 & \\
    F-00.4 & Eine Website kann     zum Aufrufen der Daten verwendet werden und dient als User Interface. & 1 & F-00.3 \\
    F-00.5 & Eine Android-App kann zum Aufrufen der Daten verwendet werden und dient als User Interface. & 3 & F-00.3 \\
    \hline


    \multicolumn{4}{|l|}{F-10 Installation und Adminstration}\\
    \hline
    F-10.1 & Die Server-Software kann mithilfe eines Installations-Skripts installiert werden. & 2 & F-00.4 \\
    F-10.2   & Ein Administrator kann die Server-Anwendung verwalten. & 1 & F-00.4\\
    F-10.2.1 & Es können Benutzer angelegt werden. & 1 & F-10.2 \\
    F-10.2.2 & Es können Benutzer gelöscht werden. & 1 & F-10.2 \\
    F-10.2.3 & Es können externe Dienste konfiguriert werden. (Siehe: F-50)& 2 & F-10.2 \\
    F-10.2.4 & Die gesammelten Daten können exportiert und importiert werden. & 2 & F-10.2 \\
    F-10.3 & Die Android-App kann aus dem Google Play Store heruntergeladen werden. & 3 & F-00.5\\
    \hline
    \multicolumn{4}{|l|}{F-20 Einträge}\\
    \hline
    F-20.1 & Ein Eintrag kann erstellt werden. & 1 & \\
    F-20.2 & Ein Eintrag kann gelöscht werden. & 1 & \\
    F-20.3   & Ein Eintrag kann bearbeitet werden. & 1 & \\
    F-20.3.1 & Komponenten können hinzugefügt, editiert und entfernt werden. & 1 & \\
    F-20.4 & Einträge anderer Benutzer können in die eigenen Einträge übernommen werden. & 1 & \\
    F-20.5 & Ein Eintrag kann mit anderen Benutzern geteilt werden. & 1 & \\
    F-20.6 & Einem Eintrag können mehrere Tags zugewiesen werden, die als Kategorien fungieren. & 1 & \\
    F-20.7 & Einträge können als privat markiert werden.            Solche Einträge werden anderen Nutzern nicht angezeigt. & 3 & \\
    F-20.8 & Einträge können als broadcast-Eintrag markiert werden. Solche Einträge werden allen Nutzern angezeigt. & 3 & \\
    \hline
    \multicolumn{4}{|l|}{F-30 Newsfeed}\\
    \hline
    F-30.1 & Der Newsfeed zeigt die Einträge des aktuellen Benutzers an. & 1 & \\
    F-30.2 & Der Newsfeed zeigt Informationen aus externen Informationsquellen an. (Siehe F-50) & 2 & \\
    F-30.3 & Der Newsfeed zeigt Einträge anderer Benutzer als Vorschläge an, zu den eigenen Einträgen hinzugefügt werden können. & 1 & \\
    F-30.3.1 & Die Auswahl der angezeigten Vorschläge erfolgt aufgrund einer Prognose der Software, wie relevant diese Einträge für den Benutzer sein könnten. Die Prognose wird auf Basis der vorhandenen Einträge des Benutzers getroffen. & 1 & \\
    F-30.4 & Alle Informationen im Newsfeed werden in einer Liste nach Relevanz sortiert angezeigt. & 1 & \\
    F-30.5 & Private Einträge werden nicht bei anderen Benutzern angezeigt. & 3 & F-20.7 \\
    F-30.6 & Broadcast-Einträge werden bei allen Nutzern angezeigt. & 3 & F-20.8 \\
    \hline
    \multicolumn{4}{|l|}{F-40 Suche}\\

    
    % ----------------------------------------- Lesezeichen: Hier habe ich aufgehört! (Please do not remove!) ------------------------------------
    % ToDo: Klassifizierung von Einträgen
    % ToDo: Broadcast-Einträge: Werden auf jedem fall allen Kursteilnehmern angezeigt
    % ToDo: 1 Installation pro Kurs
    % ToDo: Webseite: Mobil und Desktop
    % ToDo: Nutzerregistrierungsprozess
    % ToDo: Nichtfunktionale Requirements
    %           - Sicherheit
    %           - Performance


    \hline
    F-40.1 & Einträge innerhalb des Newsfeeds können durchsucht werden. & 1 & F-30\\
    F-40.2 & Es kann nach einer Kategorie gesucht werden. & 1 & F-20.7\\
    F-40.3 & Es können Einträge anhand von Benutzernamen gesucht werden. & 1 & F-10.9\\
    F-40.4 & Einträge können nach Schlüsselwörtern durchsucht werden. & 1 & \\
    F-40.5 & Einträge können anhand eines Datums durchsucht werden. Es kann ein Zeitraum angegeben werden. & 2 & \\
    F-40.6 & Zur Datums basierten suche können Alltagswörter wie z.B. \enquote{morgen}, \enquote{im Januar} verwendet werden. & 3 & F-40.5\\
    F-40.7 & Die Einträge können Anhand von Dokumenten durchsucht werden. & 2 & F-20.8\\
    \hline
    \multicolumn{4}{|l|}{F-50 Externe Dienste}\\
    \hline
    F-50.1 & Der Administrator kann externe Dienste registrieren, welche als Datenbasis zum Vorschlagen von Einträgen dienen. & 1 & \\
    F-50.2 & Es kann ein E-Mail Verteiler angegeben werden. & 1 & \\
    F-50.3 & Es kann ein online Kalender (ical) angegeben werden. & 2 & \\
    F-50.4 & Es kann eine Dropbox angegeben werden. & 2 & \\
    F-50.5 & Es kann Google Drive angegeben werden. & 3 & \\
    F-50.7 & Es kann ein ftp-Server angegeben werden. & 3 & \\
    F-50.8 & Es kann der DHBW-Mannheim Mensa Plan angegeben werden. & 3 & \\
    \hline
    \multicolumn{4}{|l|}{F-50.A Datenanalyse Externe Dienste}\\
    \hline
    F-50.A.1 & Externe Dienste werden nach Daten durchsucht aus denen neue Einträge für die Benutzer generiert werden. & 1 & \\
    F-50.A.2 & Externe Dienste werden nach einem Datum durchsucht, wodurch Einträge generiert werden welche anhand des Datums in den Newsfeed eingebettet werden können. & 1 & \\
    F-50.A.3 & Externe Dienste werden nach Links (Url) durchsucht. Dadraus wird ein Eintrag generiert der dem Benutzer diese als nützlichen Link vorschlägt. & 1 & \\
    F-50.A.4 & Externe Dienste werden nach Dokumenten untersucht, welche im Newsfeed dem Benutzer zum download angeboten werden. & 2 & \\
    F-50.A.5 & Externe Dienste werden nach User namen / E-Mail Addressen durchsucht, wodurch gezielte Eintragsvorschläge für einen Benutzer generiert werden. & 2 & F-10.9 \\
    F-50.A.6 & Externe Dienste werden nach Ortsnamen durchsucht, wodurch ein Link zu Google maps in einem Eintrag generiert wird. & 3 & \\
    \hline
    \multicolumn{4}{|l|}{F-60 Vorschläge}\\
    \hline 
    F-60.1 & Dem Benutzer werden Vorschläge anhand von Einträgen anderer Nutzer angezeigt. & 1 & \\
    F-60.2 & Dem Benutzer werden Vorschläge für Einträge anhand von externen Diensten angezeigt. & 1 & F-50 \\
    F-60.2 & Ein Benutzer kann gefragt werden ob aus den von Norbert analysierten Daten ein Eintrag angelegt werden soll. Dabei steht die Wahlfreiheit bei dem Benutzer. & 2 & \\
    F-60.3 & Im Falle einer Allgemeinen Frage, wie z.B. der Bestätigung eines Klausurtermins kann ein einzelner Benutzer die Frage für alle Benutzer beantworten. & 2 & F-60.2 \\
    F-60.4 & Einträge werden klassifiziert und bilden somit Abhängigkeiten von einander. Anhand dieser Abhängigkeiten werden einem Nutzer Vorschläge gemacht welche Einträge ihn interessieren könnten. & 2 & \\

    \hline
    \multicolumn{4}{|l|}{F-60.4 Klassifizierung von Einträgen}\\
    \hline

    \hline
    \multicolumn{4}{|l|}{Erinnerungen}\\
    \hline

    F-10 & Die Software dient als persönlicher Assistent während des Studiums. & & \\
    F-10 & Planen von Aufgaben & 1 & \\
    F-10.1 & Es können Aufgaben erstellt werden. & 1 & \\

    F-20 & Dem Benutzer werden hilfreiche Informationen über das Studium angezeigt. & 1 & \\

    F-30 & Die Software kann Erinnerungen anzeigen. & 1 & \\

    F-40 & Die Software kann Push-Notifikations an den Benutzer senden. & & \\

    F-50.2 & Es können Benutzer durch die eingabe von Emailadresse hinterlegt werden. & & \\
    F-50.2.1 & Die so erstellten Benutzer erhalten über eine Email Zugriff auf das System. & & \\ % TODO: Formulierung
    
    F-60 & Die Software kann auf Datenquellen, wie z.B. Google-Drive, DHBW Seite, Moodle etc. zugreifen. & & \\

    F-70 & Die Software kann Informationen zwischen den Studenten austauschen. & & \\
    F-70.1 & Todo's von anderen Studenten werden angezeigt. & &\\
    F-70.2 & Es können Kursnachrichten versendet werden. & &\\
    F-80 & Es können Bezüge zu Dateien aus Dropbox etc. hergestellt werden. & &\\
    F-90 & User werden anhand ihres namen markiert & &\\ %Noch unklar

    F-* & Formulare & &\\

    F-* & Todo's können privat sein. & & \\

    


    % Todo liste analysieren ob es andere gibt.
    % ToDo: Prozessdiagramm



    \bottomrule
\end{tabularx}


%Ideensammlung


%Datums-termine aus Dokumenten finden % 1
%PDFs parsen % 2
%Dokumente anhand von Keywörtern gruppieren -> Vorschlag für weitere Quellen. % 2
%Vorschlag von dokumenten anhand von tasks erstellen. % 2
%
%Emails im Kursverteiler (Samt angehängter Dokumente) parsen. % 1
%    - Sobald genug Daten im System sind: Erkennung des Fachs, zu dem die Nachricht gehört. (Absenderadresse, Signatur, Betreff, Keywords) % 2
%    - Datumsangaben erkennen --> Erinnerungen % 1
%    - Phrasen wie "bis zur nächsten Vorlesung" erkennen. Wenn das Fach erkannt wurde: Abgleich mit dem Kurskalender --> Erinnerungen % 1
%
%Erkennen von Trends (z.B.: 10 Studenten legen Aufgabe für Mathe-Hausaufgabe an)
%    --> Die restlichen Studenten bekommen solche Aufgaben als Vorschlag. % 1
%Erkennen von Nutzergruppen mit ähnlichen Aufgaben (z.B. die Mitglieder von Norbert)
%    --> Vorschläge von Aufgaben innerhalb dieser Nutzergruppen % 1
%(Beides ist für den Nutzer nicht sichtbar. für ihn wirkt es bloß "intelligent" die eigentliche 
%Funktion wäre also: "Vorschlagen von Aufgaben / Hinweisen von anderen Kursteilnehmern basierend
%auf einem personalisierten Ranking... oder so ähnlich...).
%
%Google Places anzeigen -> Pizza, Bars, Festivals, Freizeitmöglichkeiten % 3
%
%Diverse DHBW-Dienste per webscraping parsen (Stundenplan, Mensaplan, Dualis???) - Am Besten mithilfe irgend einer Plugin-Architektur, damit dass System auch bei anderen UNIs laufen kann.
%    - Immer Ende der aktuellen Vorlesung bzw. Beginn der nächsten Vorlesung anzeigen.
%
%    %Stundenplan 1
%    %Mensaplan 1
%    %Dualis 3
%
%Export der Erinnerungen als ical --> Macht synchronisation mit Google kalendar / Apple kalender / Outlook / ... Möglich % 2
%
%Android APP: % 3
%    - Erinnerungen
%    - Wecker, der immer HH:MM vor Vorlesungsbeginn klingelt. (Oder: Systemwecker immer richtig einstellen)
%
%Fächer % 1 
%    - Bei SEEEEEHR vielen Stellen hilft es, wenn man die Einträge einem Fach zuordnen kann. (Bessere Vorschläge oder Vorschläge so überhaupt erst realisierbar). Daher sollte hier eine globale Liste aller Fächer existieren. Der Admin kann die Liste bei der Installation "initial" mit Werten füllen, aber auch Studenten sollten Fächer hinzufügen können)
%    - Feld für das Fach sollte jedoch nie erzwungen werden, um es dem Benutzer so einfach wie möglich zu machen. Wenn das feld freigelassen wurde: 
%    - Zuordnung zum Fach bei bestimmten (gelernten?) Schlüsselwärtern automatisch. Insbesondere natürlich, wenn in einem anderen Feld der Name eines Fachs erwähnt wurde (z.B.: Titel: "Datenbanken lernen" --> Fach ist offensichtlich Datenbanken. Oder: "Mathe lernen" --> Fach ist statistik, da das wort "Mathe" bei anderen Einträgen oft mit dem Fach Statistik in Verbindung gebracht wurde.)
%    - (Für den Benutzer könnten die Fächer evtl. wie Tags angezeigt werden)
%    - Die Zuordnung Fach <--> Professor ist auch sehr hilfreich, um die Vorlesungsmaterialien, die über den Kurskalender reinkommen, einem Fach zuordnen zu können (--> Durchsuchbarkeit!). Ich würde versuchen, diese Zuordnung zu lernen. Alternativ wäre auch das eine Datenquelle, für die sich eine manuelle Eingabe vom Aufwand/Nutzen-Verhältniss lohnen würde.
%
%ZIELE und HERLEITUNG VON ZEUGS VON DEN ZIELEN
% - Ein erfüllteres Studentenleben
%    - Bessere Noten
%        - Gezielter lernen
%            - Lernplan kann erstellt werden
%            - Umsetzung Über ToDos, die Unterpunkte haben können. (Kann natürlich auch für anderes genutzt werden)
%            - Vorschläge für Unterpunkte aus ähnlichen Lernplänen anderer Kursteilnehmer. --> So wird auf jedem Fall nichts vergessen.
%            - Als "Ähnlichkeitskriterium" ist auch hier das Fach sehr wichtig...
%        - Mehr lernen
%            - Über die Vorschlagsfunktion
%            - (Viele Studenten legen Lerntasks an --> Man selber wird gefragt, ob man nicht lernen möchte)
%        - Früh genug anfangen zu lernen.
%            - Siehe Punkt "Mehr lernen"
%        - Motivation zum Lernen
%            - Gamification...??????????
%            - Coole Animation beim Abhaken von erledigten Tasks.
%            - Lobende Worte beim Abhaken erledigter Tasks.
%            - Möglichkeit zum Teilen erledigter Tasks.
%        - Unterstützung beim Lernen
%            - Fragen stellen, Experten zu Themen finden.
%                - Naja, Eigentlich sind wir ja kein Forum...
%            - Lerngruppen
%                - Vorschlag, Lerngruppen zu bilden
%                    - 2 Benutzer haben eine ähnliche Erinnerung zu ähnlichen Zeiten angelegt
%                    - Aufgrund der Analyse von sozialen Beziehungen im Kurs (Wer weist wem wie häufig Tasks zu, ...)
%                - Möglichkeiten zu Lerngruppenanfragen
%                    --> 1 Benutzer erstellt Lerngruppenanfrage.
%                    --> Anfrage erscheint beim ganzen Kurs ganz oben im Newsfeed
%                    --> Auswahlmöglichkeiten: (Y) Accept - (N) Decline
%                    --> Möglichkeit zur Angabe von Präferenzen, mit wem man gerne lernen will und wie groß die eigene Lerngruppe sein soll.
%                    --> Es erfolgt ein automatisches Matching der Lerngruppen unter Berücksichtigung der Präferenzen der user.
%                    --> Anschließend kann natürlich noch manuell aus lerngruppen ausgetreten werden bzw. Personen hinzugefügt werden.
%                    --> Lerngruppen verweilen im Newsfeed der Teilnehmer. Innerhalb der Lerngruppen können Erinnerungen angelegt werden, die für alle Teilnehmer gelten.
%                    --> Evtl eine Möglichkeit zum Erstellen einer Telegram/Whattsapp-Gruppe mit einem Klick.
%            - Lernmaterialien
%                - Literatur zum lernen, die die komplette Vorlesung umfasst
%                        - Basierend auf
%                            - Literaturangaben/Empfehlungen in Dokumenten des Professors
%                            - Ähnlichkeit des Inhaltsverzeichnis mit Dokumenten (Slides...) des Professors
%                            - Bücher, die der Professor selbst geschrieben hat, oder von denen er die Autorennamen in einem Dokument erwähnt hat.
%                            - Interaktion anderer Studenten mit den Vorschlägen (Bei wie vielen anderen Studenten wurde der Vorschlag weggewischt bzw. angeklickt)
%                        - Datenquellen
%                            - Springer Link, andere EBook-Seiten (Fokus auf Seiten, die für den Student kostenlos sind)
%                            - Online-Katalog v. DHBW-Bib (und andere, am Besten locationbasiert oder Pluginsystem)
%                            - (Lokale) Buchläden (Location!)
%                            - Amazon-Links zu Büchern (Am Besten Affiliate-Links...!)
%                            - Vorlesungsskripte von anderen Universitäten
%                            - Vorlesungsaufzeichnungen von anderen Universitäten
%                        - Solche Vorschläge bevorzugt am Anfang des Semesters und vor den Klausurphasen (Lernphasen könnte man anhand der vom Benutzer angelegten Tasks erkennen!)
%                - Material zum besseren Verständnis einzelner Themen.
%                    - Basierend auf
%                        - Überschriften/Schlagwörter in den Materialien der Dozenten, die per Kursverteiler reinkommen.
%                    - Datenquellen
%                        - Links in den Vorlesungsunterlagen der Professoren (oder andere Dokumente im System)
%                        - Wikipedia-Artikel zu Themen aus den Vorlesungsunterlagen / Anderen Dokumenten
%                        - Online-Kurse??? (Coursera, etc...) ??????????????????????????????????
%                        - Google-Suchergebnisse (Wie Qualität sicherstellen???)
%                    - Solche Vorschläge bevorzugt direkt nach einer Vorlesung zum entsprechenden Thema.
%                - Vorlesungsunterlagen
%                    - Durchsuchbar
%                        - Über unser Tool kann nach Stichworten im Text (innerhalb des Dokuments) gesucht werden und nach Fach gefiltert werden.
%                    - Zentral abrufbar
%                        - Vorlesungsmaterialien, die per Mail reinkommen können automatisiert in ein "Dokumentenmanagementsystem" wie Dropbox importiert werden. Selbstverständlich ordentlich nach Fach und Datum sortiert.
%                    - Übersichtlich sortiert
%                    - Sollte auch aktiv vom Student zur Vor-/Nachbereitung genutzt werden.
%                        - Vorschläge zum Wiederholen der Vorlesungsinhalte, wenn Materialien verfügbar.
%    - Stressfreier Alltag
%        - Spaß haben
%            - Freunde finden
%            - Freizeitaktivitäten
%                - Google Places
%                - Facebook-Events in der Umgebung
%                - Schneckenhof-Parties
%                - Vorschläge aus dem Angebot der Uni (Sport, Kursverteilerevents wie Kneipentouren)
%            - Mehr Freizeit
%                - Nicht zu viel lernen. Bzw. effizienter lernen.
%        - Stressfreie Vorlesungen
%            - Vorhersage des DHBW-Glücksrads
%                -> Phillip fragen, ob er für uns den Code nicht noch mal etwas abändern will ;-)
%                -> Vieleicht auch nicht...
%        - Informiert sein
%            - Tages-Briefing
%                - Heutiger Stundenplan
%                - Wann muss ich losfahren, um rechtzeitig anzukommen (Auto, Bahn)
%                - Was gibt es in der Mensa
%                - Anstehende Aufgaben
%                - Wann beginnt die morgige Vorlesung
%        - Weniger DHBW-Bürokratie
%            - Erinnerung an Fristen
%                - Parsen der Dokumente/Kursverteilermails auf Datumsangaben. --> Erinnerung mit der entsprechenden Textpassage
%                - Es können auch händisch kursweite Erinnerungen angelegt werden.
%            - Hinweise bei Stundenplan / Raumänderungen
%                --> Parsen des Online-Stundenplans
%                --> Per Pluginsystem etc, damit auch für andere UNIs einsetzbar
%            - Raussuchen der richtigen Dokumente und wo möglich teilweise automatische Ausfüllung
%                - Keine Ahnung, wie das ohne aufwändige manuelle Eingaben umsetzbar wäre.
%                - Höchstens: Wenn in Emails auf Dokumente verwiesen wird.
%        - Kursorganisation
%            - Globale Hinweise, manuell erstellt
%            - Globale Erinnerungen, manuell erstellt (z.B. an Fristen, ...)
%            - Globale Abstimmungen
%
%
%
%Moodle, Dropbox:
%    Überall wo ich "Kursverteiler" geschrieben habe, wird natürlich auch Moodle und Dropbox sowie alle anderen konfigurierten Datenquellen impliziert...
%
%Grobe Aufgaben
%    1) Sammeln von eingehenden Kurs-Daten
%        - Emailverteiler
%        - Dropbox: Hochgeladene Dokumente
%        - Moodle
%        - ...
%    2) Klassifizierung der Daten
%        - Fach
%        - Oder Fächerübergreifend (Organisationsmails von Studienleitung)
%        - Themen und Schlagwörter
%        - Offizielles Vorlesungsmaterial oder unterstützende Materialien von Kursteilnehmern?
%    3) Indizierung für die Suche
%    4) Analyse auf Datumsangaben (Hinweis auf mögliche Deadlines)
%    5) Speicherung
%        - Dokumente > Dropbox
%        - Nachrichten > Hinweis
%        - Nachrichten mit Deadlines > Vorschlag zum Anlegen einer Erinnerung
%    6) Möglichkeit zur manuellen Erzeugung von Datensätzen
%        - Hinweise
%        - Erinnerungen
%        - Aufgaben
%        Anhängen von Dokumenten möglich --> Dropbox
%    7) Analyse von Ähnlichkeiten zwischen den Datensätzen
%    8) Erzeugung von Vorschlägen basierend auf den Ähnlichkeiten.
%    9) Aggregierung/Indizierung von Daten im Internet (APIs oder Webscraping)
%        - Stundenplan
%        - Mensaseite
%        - Facebook Events 
%        - Google Places
%        - Springer Link
%        - Onlinekatalog bib
%        - ...
%
%
