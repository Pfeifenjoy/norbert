%!TEX root = ../documentation.tex

\chapter{Konventionen}
\section{Identifizieren von Aufgaben, Funktionen und Eigenschaften}
In diesem Dokument werden Aufgaben oder Eigenschaften mit einer ID identifiziert.
Dabei spaltet sich jede ID in einen Buchstaben, der für das Kapitel steht und eine Nummer, die für das Unterkapitel steht.\\
\begin{center}
    \textbf{Bsp.: A-10.1}
\end{center}
Eine genaue Übersicht zu den Kapiteln ist in der Tabelle \ref{einleitung:kapitel} zu finden.
\begin{table}[H]
    \caption{ID Kapitel Zuweisung}
    \label{einleitung:kapitel}
    \begin{tabularx}{\textwidth}{|l|X|}
        \toprule
        \textbf{ID-Kürzel} & \textbf{Kapitel} \\
        \endhead
        \hline
        F & Funktionen \\
        D & Daten \\
        SE & Softwareumgebung \\
        Q & Qualitätsziele \\
        T & Testszenarien \\
        \hline
    \end{tabularx}
\end{table}

\section{Prioritäten}
In diesem Dokument wird die Priorität der Funktionen durch eine Nummer zwischen 1 und 3 angegebenen. Nähere Informationen finden sich in der nachfolgenden Tabelle \ref{einleitung:priority}.
\begin{table}[H]
    \caption{Prioritätsskala}
    \label{einleitung:priority}
    \begin{tabularx}{\textwidth}{|l|X|}
        \toprule
        \textbf{Nummer} & \textbf{Wertigkeit}\\
        \endhead
        \hline
        1 & Muss-Kriterium\\
        2 & Soll-Kriterium\\
        3 & optionale Umsetzung\\
        \hline
    \end{tabularx}
\end{table}
