%!TEX root = ../documentation.tex

\chapter{Ziele}
Das Ziel des Projektes ist es, eine Software zu entwickeln, mit der sich der Studienalltag von DHBW-Studenten einfacher gestalten lässt. Die Software wird von den Studenten genutzt um  Aufgaben zu planen und zu verwalten, Errinerungen zu erhalten und Dokumente auszutauschen. Zusätzlich liefert die Software Vorschläge zu Aufgaben und ToDo's, sowie Informationen, die den Studenten interessieren könnten, zum Beispiel den Mensaplan. Das primäre Ziel ist, die Planung von Aufgaben und ToDo's einfacher und effektiver zu gestalten. Dabei soll die Software dem Studierenden Arbeit ersparen und Zeit effektiver nutzen.

\begin{table}[H]
\caption{Ziele}
\label{ziele:entwicklungsziele}
\begin{tabularx}{\textwidth}{|l|X|l|}
\toprule
\textbf{ID} & \textbf{Beschreibung} & \textbf{Priorität}\\
\endhead
\hline
Z-10 & Den Studienalltag einfacher gestalten. & 1 \\
Z-20 & StudiLife-Leser für ein DHBW-Studium begeistern & 1 \\
Z-20.1 & Vorteile der DHBW vermitteln. & 1 \\
Z-20.2 & Vorteile des dualen Systems vermitteln. & 1 \\
Z-30 & Das Produkt ist eine spielbare Software. & 1\\
\hline
\end{tabularx}
\end{table}
