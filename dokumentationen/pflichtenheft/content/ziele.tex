%!TEX root = ../documentation.tex

\chapter{Ziele}
Das Ziel des Projektes ist es, eine Software zu entwickeln, mit der sich der Studienalltag von DHBW-Studenten einfacher gestalten lässt. Dabei soll die Software dem Studierenden Arbeit ersparen und helfen, Zeit effektiver zu nutzen. Die Software wird von den Studenten genutzt, um  Aufgaben zu planen und zu verwalten, Errinerungen zu erhalten und Dokumente auszutauschen. Zusätzlich liefert die Software Vorschläge zu Aufgaben und ToDo's, sowie Informationen, die den Studenten interessieren könnten, zum Beispiel den Mensaplan. 

\begin{table}[H]
\caption{Ziele}
\label{ziele:entwicklungsziele}
\begin{tabularx}{\textwidth}{|l|X|l|}
\toprule
\textbf{ID} & \textbf{Beschreibung} & \textbf{Priorität}\\
\endhead
\hline
Z-10 & Die Software hilft, den Studienalltag von DHBW-Studenten einfacher zu gestalten. & 1 \\
Z-20 & Das Produkt kann zur Planung und Verwaltung von Aufgaben und ToDo's verwendet werden. & 1 \\
Z-30 & Das Produkt erinnert an Fristen und andere Termine  & 1 \\
Z-40 & Über das Produkt lassen sich Dokumente austauschen. & 1 \\
Z-50 & Das Produkt liefert Vorschläge zu Aufgaben und ToDo's & 1 \\
Z-60 & Das Produkt liefert Informationen, die für den Studierenden interessant sein könnten. & 1 \\
Z-70 & Das Produkt ist eine über das Internet nutzbare Software. & 1\\
\hline
\end{tabularx}
\end{table}
