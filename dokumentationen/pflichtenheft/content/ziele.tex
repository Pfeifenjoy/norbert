%!TEX root = ../documentation.tex

\chapter{Ziele}
Das Ziel des Projektes ist es, eine Software zu entwickeln, mit der sich der Studienalltag von DHBW-Studenten einfacher gestalten lässt. Dabei soll die Software dem Studierenden Arbeit ersparen und helfen, Zeit effektiver zu nutzen. Die Software wird von den Studierenden genutzt, um  Aufgaben zu planen und zu verwalten, Errinerungen zu erhalten und Dokumente auszutauschen. Zusätzlich liefert die Software Vorschläge zu Aufgaben und ToDo's, sowie Informationen, die die Studierenden interessieren könnten, zum Beispiel den Mensaplan. 

\begin{table}[H]
\caption{Ziele}
\label{ziele:entwicklungsziele}
\begin{tabularx}{\textwidth}{|l|X|l|}
\toprule
\textbf{ID} & \textbf{Beschreibung} & \textbf{Priorität}\\
\endhead
\hline
Z-10 & Die Software hilft, den Studienalltag von DHBW-Studierenden einfacher zu gestalten. & 1 \\
Z-20 & Die Software erleichtert den Erfahrungsaustausch zwischen Studenten. & 1 \\
Z-30 & Die Software hilft, Zeit effektiver zu nutzen. & 1\\
Z-40 & Die Software ist am PC und am Smartphone nutzbar. & 1\\
\hline
\end{tabularx}
\end{table}
