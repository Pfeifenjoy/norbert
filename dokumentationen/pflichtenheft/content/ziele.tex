%!TEX root = ../documentation.tex

\chapter{Ziele}
Das Ziel des Projektes ist es, eine Software zu entwickeln, mit der sich der Studienalltag von DHBW-Studenten einfacher gestalten lässt. Dabei soll die Software dem Studierenden Arbeit ersparen und helfen, Zeit effektiver zu nutzen. Die Software wird von den Studierenden genutzt, um  Aufgaben zu planen und zu verwalten, Errinerungen zu erhalten und Dokumente auszutauschen. Zusätzlich liefert die Software Vorschläge zu Aufgaben und ToDo's, sowie Informationen, die die Studierenden interessieren könnten. 

\section{Was soll Norbert leisten?}
\begin{table}[H]
\caption{Ziele}
\label{ziele:entwicklungsziele}
\begin{tabularx}{\textwidth}{|l|X|l|}
\toprule
\textbf{ID} & \textbf{Beschreibung} & \textbf{Priorität}\\
\endhead
\hline
Z-10 & Die Software hilft, den Studienalltag von DHBW-Studierenden einfacher zu gestalten. & 1 \\
Z-20 & Die Software erleichtert den Erfahrungsaustausch zwischen Studenten. (Wissensmanagement \& Wissensweitergabe) & 1 \\
Z-30 & Die Software hilft, Zeit effektiver zu nutzen. (Zeitmanagement) & 1\\
Z-40 & Die Software ist am PC und am Smartphone nutzbar. & 1\\
Z-50 & Ziel ist es, dass der Benutzer eine kurze Lernphase zur Benutzung der Software hat. & 1 \\
Z-60 & Die Software soll möglichst leicht zu installieren und zu warten sein. & 1 \\
Z-70 & Die Informationen von mehreren Diensten wie z.B. E-Mail Verteiler oder Dokumentenmanagement Systemen sollen dem Studenten übersichtlich presentiert werden. Der Student hat eine Einheitliche möglichkeit diese Dienste zu durchsuchen. & 2 \\
Z-80 & Durch die Bündelung des Wissens der Studenten soll verhindert werden, dass Studenten vergessen Aufgaben zu erledigen. & 2 \\
Z-90 & Es soll Spaß machen die Software zu verwenden. & 2 \\
Z-100 & Der Student soll bei seinem Studium unterstütz werden, aber immernoch selbst Entscheidungen treffen. & 1 \\
\hline
\end{tabularx}
\end{table}

\section{Was soll Norbert nicht sein?}

Norbert soll keine allgemeine Lösung zum Verwalten von Dokumenten und Vorlesungsinhalten wie z.B. Moodle sein.
Es wird nicht der Fokus darauf gelegt durch viele Funktionen und komplexe Workflows den Studienalltag zu managen,
vielmehr durch Filtern der Informationen aus Dokumenten oder E-Mails etc. dem Student eine Übersicht zu seinen aktuellen Aufgaben während des Studiums zu geben.
Dabei ist Norbert keine Lösung zum Selbstmanagement, sondern dient zur Bewältigung von Problemen und Aufgaben in einem Kurs als Ganzes. 
Dadurch soll es ermöglicht werden Probleme und Aufgaben die alle betreffen oder betreffen könnten im Kurs zu verteilen.
Im Bezug auf Punkt Z-100 kann unsere Software dem Student nicht alle Aufgaben des Studiums präsentieren. Die Verantwortung \textbf{alle} Aufgaben / Dokumente zu managen liegt immernoch bei den Studierenden. 
