%!TEX root = ../documentation.tex

\chapter{Daten}

In diesem Kapitel wird beschrieben, welche Daten für die Anwendung unserer Software relevant sind.\\
\begin{longtable}{|l|p{5cm}|l|}
    \toprule
    \textbf{ID-Kürzel} & \textbf{Beschreibung} & \textbf{Priorität}\\
    \hline
    \endhead
    \hline
    \caption{Daten}
    \endfoot

    D-10 & Nutzerspezifische Daten & 1\\
    D-10.1 & Name &1\\
    D-10.2 & E-Mail-Adresse &1 \\       
    \hline
    D-20 & Eintragsbezogene Daten & 1\\
    D-20.1 & Titel &1\\
    D-20.2 & Beschreibung & 1\\
    D-20.3 & Zugewiesene Personen & 1\\
    D-20.4 & Erinnerung &1\\
    D-20.5 & Link &1\\
    D-20.6 & Anhang &1\\
    D-20.7 & Aufgaben & 1\\
    D-20.8 & Zeitstempel & 1 \\
    D-20.9 & Ort &  2\\
    
    \hline
    D-40 & Externe Daten & 2\\
    D-40.1 & E-Mail-Verteiler & 2\\
    D-40.2 & Externe Webseiten & 3 \\
    D-40.2.5 & Moodle & 3\\        
    D-40.2.1 & Online Bibliothek & 3\\
    D-40.2.2 & Dropbox & 3 \\
    D-40.2.3 & Bahnverbindungen & 3\\
    D-40.2.4 & Menü der Mensa & 3\\
    D-40.2.5 & Doodle & 3\\        
    \hline
\end{longtable}


