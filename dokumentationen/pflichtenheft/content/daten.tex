%!TEX root = ../documentation.tex

\chapter{Daten}

In diesem Kapitel wird beschrieben, welche Daten für die Anwendung unserer Software relevant sind.\\

\begin{table}[H]
    \caption{Daten}
    \label{einleitung:kapitel}
    \begin{tabularx}{\textwidth}{|l|X|}
        \toprule
        \textbf{ID-Kürzel} & \textbf{Kapitel} \\
        \endhead
        \hline
        D-10 & Nutzerspezifische Daten \\
        D-10.1 & Benutzername \\
        D-10.2 & E-Mail-Adresse \\
        D-10.3 & DHBW-Kurskürzel\\
        \hline
        D-20 & Aufgabenbezogene Daten \\
        D-20.1 & Aufgabenbeschreibung \\
        D-20.2 & Zugewiesene Personen \\
        D-20.3 & Fertigstellungsdatum \\
        \hline        
        D-30 & Erinnerungs- und Termindaten \\
        \hline
        D-40 & Externe Daten \\
        D-40.1 & E-Mail-Verteiler \\
        D-40.2 & Externe Webseiten \\
        D-40.3 & Dokumente (Bsp. PDF)\\
        \hline
    \end{tabularx}
\end{table}


