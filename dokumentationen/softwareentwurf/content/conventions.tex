%!TEX root = ../documentation.tex


\chapter{Code Conventions}
Im folgenden werden die Code Conventions genannt, die während des Projekts eingehalten werden sollen, um einheitliche Code zu erzeugen.


\begin{tabularx}{\textwidth}{|l|X|X|}
    \toprule
    \textbf{ID-Kürzel} & \textbf{Beschreibung} & \textbf{Begründung}\\
    \midrule
    \endhead
    \hline
    \caption{Code Conventions}
    
    \endfoot
    C-10 & englische Variablennamen &  Es gibt sehr viel mehr englische Dokumentationen zu den einzelnen Programmiersprachen. Damit nicht immer ein Umdenken bzw. Übersetzen stattfinden muss, ist es einfacher die englische Sprache zu verwenden.\\
    C-20 & englische Kommentare & Damit der ganze Quelltext in einer Sprache verfasst ist, wird auch hier Englisch als Sprache verwendet.  \\
    C-30 & Module & Voneinander unabhängige Programmteile (z.B. Klasse, Funktionen) werden getrennt, um die Wartbarkeit zu verbessern und eine bessere Übersicht zu gewährleisten.  \\
    C-40 & Lokale Variablen & Falls immer es möglich ist, sollen lokale Variablen (Schlüsselwort: var) verwendet werden. Dies vermeidet Änderungen von außerhalb am Wert der Variable.\\ 
    C-50 & Wiederverwendbarkeit & Falls Funktionen wiederverwendet werden können, wird dies getan, um Redundanz zu vermeiden und die Wartbarkeit des Codes zu steigern.
  
\end{tabularx}



