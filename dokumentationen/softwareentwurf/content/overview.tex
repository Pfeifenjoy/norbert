
\chapter{Grundlegende Komponenten und deren Funktionsweise}

Folgende Funktionalitäten müssen von der Architektur zwingend erfüllt werden, damit Norbert korrekt funktionieren kann:
\begin{enumerate}
	\item Der Newsfeed muss auf dem Client angezeigt werden können.
	\item Norbert muss die gesamten Daten (Einträge, Vorschläge, Emails. etc.) zentral speichern.
	\item Die für den Newsfeeds relevanten Daten müssen von Norbert an die Clients geliefert werden können.
	\item Der Client muss Einträge anlegen/bearbeiten können und auf Vorschläge reagieren können, bzw. seine Aktionen Norbert mitteilen können.
	\item Norbert muss Vorschläge liefern können.
\end{enumerate}
 


\section{Grundlegender Aufbau aus Funktionssicht}


  Die in Grafik \ref{fig: Overview_HLVL} gezeigte Architektur würde die oben genannten Kriterien erfüllen.
  \begin{enumerate}
  	\item Über einen Webbrowser kann der Newsfeed auf den Clients angezeigt werden.
  	\item Norbert hat die gesamten Daten in getrennten Dateien/Ordnern zur Verfügung.
  	\item Über die Kommunikationsschnittstelle kann Norbert die relevanten Daten an den Client liefern.
  	\item Über die Kommunikationsschnittstelle können die Aktionen der Clients Norbert mitgeteilt werden.
  	\item Die Vorschlagslogik berechnet anhand der vorhandenen Daten Vorschläge, welche über die Kommunikationsschnittstelle an den Client geliefert werden.
  	\end{enumerate}

  	Mit dieser Architektur treten aber mehrere Probleme auf:
  	\begin{enumerate}
  	\item Das manuelle Verwalten von Daten in separaten Dateien ist sehr aufwendig und extremst fehleranfällig.
  	\item Der genaue Ablauf 
  	\end{enumerate}
\begin{figure}[H]
\centering
\includegraphics[scale=0.6]{uml-diagramms/overview_hlvl.pdf}
\caption{Grundlegende Funktionalität}
\label{fig: Overview_HLVL}
\end{figure}

\section{Datenhaltung}

\begin{figure}[H]
\centering
\includegraphics[scale=0.75]{uml-diagramms/daten_hlvl.pdf}
\caption{Kapselung der Daten in einer Datenbank}
\label{fig: DB_HLVL}
\end{figure}

\section{Interne Prozesse}

\begin{figure}[H]
\centering
\includegraphics[scale=0.6]{uml-diagramms/overview_detail.pdf}
\caption{Kapselung der internen Prozesse}
\label{fig: Overview_Detail}
\end{figure}




        