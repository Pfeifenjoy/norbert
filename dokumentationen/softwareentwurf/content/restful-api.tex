\chapter{RESTful API}
Die RESTful API basiert auf JSON und ist die zentrale Schnittstelle zwischen den Client Anwendungen (frontend) und dem Backend.
Sie basiert auf HTTP-Requests, weshalb alle Anfragen vom Client gesteuert werden.

\section{Übersicht der Funktionen}
Im folgenden werden alle Funktionen der RESTful Schnittstelle aufgelisted. 
\begin{tabularx}{\textwidth}{|l|l|l|X|}
    \toprule
    \textbf{ID} & \textbf{Methode} & \textbf{Pfad} & \textbf{Beschreibung} \\
    \midrule
    \endhead
    \hline
    \caption{RESTful Pfade}
    \label{RESTful:pfade}
    \endfoot

    \multicolumn{4}{|l|}{R-10 Userverwaltung}\\
    \hline

    R-10.1 & POST & /users/ & Erstelle einen Benutzer. Es müssen ein \enquote{username} und ein \enquote{password} übergeben werden.\\
    R-10.2 & PUT & /users/:userId & Bearbeiten eines users.\\
    R-10.3 & DELETE & /users/:userId & Lösche einen User.\\
    R-10.4 & GET & /users/login & Authentifizierung durch session. \enquote{username} und \enquote{password} müssen übergeben werden.\\
    R-10.5 & GET & /users/logout & Beendigung der Session.\\

    \hline
    \multicolumn{4}{|l|}{R-20 Newsfeed}\\
    \hline

    R-20.1 & GET & /newsfeed/:userid & Hole alle Einträge für den Nutzer basierten Newsfeed. \enquote{top} definiert den Anfang der Einträge relativ zum aktuellen Zeitpunkt. \enquote{skip} definiert wie viele Einträge zurückgegeben werden. Zurückgegeben wird ein Array aus Einträgen.\\ 
    R-20.2 & POST & /entry/ & Füge einen Eintrag zum Newsfeed hinzu. Der Parameter heißt \enquote{entry}.\\ 
    R-20.3 & GET & /entry/:id & Gibt alle Daten zu dem Eintrag mit der id wieder.\\
    R-20.4 & DELETE & /entry/:id & Löschen eines Eintrags.\\
    R-20.5 & PUT & /entry/:id & Updaten eines Eintrags.\\

    \hline
    \multicolumn{4}{|l|}{R-30 Notifikationen}\\
    \hline
    R-30.1 & GET & /notifications/:userid/ & Holen aller Notifikationen.\\ 

    \hline
    \multicolumn{4}{|l|}{R-40 Suche}\\
    \hline
    R-40.1 & GET & /search/ & Es werden alle Einträge anhand des Parameters \enquote{query} durchsucht.
                                Das Ergebnis ist eine Liste von Einträge und Informationen zurückgegeben.\\

\end{tabularx}

\section{Authentifizierung}
Ein Benutzer darf lediglich die zu ihm gehörenden Einträge bearbeiten. 
Bei allen Funktionen außer dem Login und der Registrierung muss der Benutzer authentifiziert sein.
Die Authentifiszierung wird durch Sessions realisiert.

\section{Einträge}
Einträge werden durch die RESTful API als JSON-Objekt zurückgegeben.

Sie können folgende Struktur haben:

\begin{lstlisting}[language=JSON]
{
    "title": "Eintrag",
    "tags": ["tag1", "tag2"],
    "components": [...(Objects)]
}
\end{lstlisting}

\section{Notifikationen}
Eine Notifikation ist ein JSON-Objekt mit folgender Struktur:

\begin{lstlisting}[language=JSON]
{
    "date": int (UNIX timestamp),
    "entryId": int
}
\end{lstlisting}

Mit dieser Information kann das Frontend die Notifikation zu einem bestimmten Zeitpunkt aufrufen. Über die \enquote{entryId} kann der passende Eintrag angefordert werden.

