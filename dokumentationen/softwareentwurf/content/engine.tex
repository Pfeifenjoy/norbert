\chapter{Systemarchitektur}
Die Anwendung kann allgemein in drei große Komponenten zerlegt werden: Das Frontend, ein Backend-Service und eine Engine.

Das Frontend ist eine HTML5 Applikation welche Daten aus dem Backend-Service abrufen und anzeigen kann.

Der Backend-Service bildet eine Schnittstelle zu den Daten und kümmert sich um die Authentifizierung der Benutzer.
Zudem startet er rechenintensive Prozesse in der Engine.

In der Engine versteckt sich die Logik der Anwendung. Hier werden Daten aus der Datenbank oder einem externen Service abgerufen und \enquote{interpretiert}.
So werden z.B. E-Mails eingelesen zu einem Eintrag konvertiert welcher dann in der Datenbank gespeichert wird.
Zusätzlich klassifiziert die Engine diese Einträge. Anhand dieser Klassifizierung können dem Benutzer vorschläge für weiter Einträge gemacht werden.

\chapter{Backend}

\section{Engine}

\section{Backend-Service}


\chapter{Frontend}

Ein Ziel ist es das mehrer Objekte im Frontend sich einen Zustand teilen können. So soll z.B. die Detailansicht eines Eintrags den selben Zustand haben wie
die Anzeige des Eintrags im Newsfeed.
Des weiteren soll die Anwendung im Team stattfinden, weshalb Einzelteile der Anwedung unabhängig von einander entwickelt werden sollen.

Deshalb haben wir uns dafür entschieden React.js als Frontend Framework zu verwenden und dem Flux-Pattern zu folgen.
Im folgenden soll die allgemeine Architektur beschrieben werden.

Im allgemeinen besteht das Frontend aus Komponenten. Eine Komponente kann z.B. ein Button oder ein Formular sein.
Diese Komponenten können aus mehreren Unterkomponenten bestehen, wie z.B. ein Button teilkomponente eines Formulars sein kann.

Eine Komponenten kümmert sich um die Darstellung von Daten. Zudem hat eine Komponente immer einen veränderbaren Zustand. 
Wenn sich dieser Zustand ändert wird die Komponente neu gerendert und angezeigt.
