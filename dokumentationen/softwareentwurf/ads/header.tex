%!TEX root = ../dokumentation.tex

\documentclass[%
	pdftex,
	oneside,			% Einseitiger Druck.
	11pt,				% Schriftgroesse
	parskip=half,		% Halbe Zeile Abstand zwischen Absätzen.
	headsepline,		% Linie nach Kopfzeile.
	footsepline,		% Linie vor Fusszeile.
	headings=optiontohead,
	plainfootsepline
]{scrreprt}

%% Schrift
\usepackage{xstring}
\usepackage[utf8]{inputenc}
\usepackage[T1]{fontenc}
\usepackage{lmodern}
\renewcommand{\familydefault}{\sfdefault}

\usepackage[english,german]{babel} 

%% Mathepakete & Schrift
\usepackage{amsmath}
\usepackage{amsthm}
\usepackage{amsbsy}
\usepackage{amssymb}
\usepackage{sansmath}
\sansmath

%% Für Tables/Images/Figures
\usepackage{here}

%% ---------------------- Allgemeine Einstellungen ----------------------

\newcommand{\spaltenabstand}{10pt}
\newcommand{\zeilenabstand}{1.5}

%%%%%%%%%%%%%%%%%%%%%%%%%%%%%%%%%%%%%%%%%%%%%%%%%%%%%%%%%%%%%%%%%%%%%%%%%%%%%%%%


%%%%%%% Package Includes %%%%%%%

\usepackage[a4paper, left=25mm,right=25mm, top=38mm, bottom=43mm]{geometry}
\usepackage[onehalfspacing]{setspace}
\usepackage{makeidx}
\usepackage[autostyle=true,german=quotes]{csquotes}
\usepackage{longtable}
\usepackage{graphicx}
\usepackage{calc}		%zum Rechnen (Bildtabelle in Deckblatt)
\usepackage[right]{eurosym}
\usepackage{nameref}
\usepackage{tabularx}
\usepackage{ltablex}
\usepackage{booktabs}
\usepackage{pgfplots}
%\pgfplotsset{compat=1.13}
\usepackage{pstricks}

%% ---------------------- Verschiedenes ----------------------

%% Paket um Textteile drehen zu können
\usepackage{rotating}

%% Paket um Seite im Querformat anzuzeigen
\usepackage{lscape}

%% Farben (Angabe in HTML-Notation mit großen Buchstaben)
\usepackage{xcolor} 	%xcolor für HTML-Notation
\definecolor{LinkColor}{HTML}{00007A}
\definecolor{ListingBackground}{HTML}{FCF7DE}

\usepackage{textcomp}

%%%%%% Configuration %%%%%
% PDF Einstellungen
\usepackage[%
	pdftitle={Softwareentwurf},
	pdfauthor={Team Norbert},
	pdfsubject={Report},
	pdfcreator={pdflatex, LaTeX with KOMA-Script},
	pdfpagemode=UseOutlines, 		% Beim Oeffnen Inhaltsverzeichnis anzeigen
	pdfdisplaydoctitle=true, 		% Dokumenttitel statt Dateiname anzeigen.
	pdflang={de}, 			% Sprache des Dokuments.
]{hyperref}

% (Farb-)einstellungen für die Links im PDF
\hypersetup{%
	colorlinks=true, 		% Aktivieren von farbigen Links im Dokument
	linkcolor=LinkColor, 	% Farbe festlegen
	citecolor=LinkColor,
	filecolor=LinkColor,
	menucolor=LinkColor,
	urlcolor=LinkColor,
	linktocpage=true, 		% Nicht der Text sondern die Seitenzahlen in Verzeichnissen klickbar
	bookmarksnumbered=true 	% Überschriftsnummerierung im PDF Inhalt anzeigen.
}
% Workaround um Fehler in Hyperref, muss hier stehen bleiben
\usepackage{bookmark} %nur ein latex-Durchlauf für die Aktualisierung von Verzeichnissen nötig

% Schriftart in Captions etwas kleiner
\addtokomafont{caption}{\small}

\setlength{\tabcolsep}{\spaltenabstand}
\renewcommand{\arraystretch}{\zeilenabstand}


%% Header und Footer
\usepackage[autooneside=false,automark]{scrlayer-scrpage}

 
% Löschen der Kopfzeileneinträge bei neuem Kapitel;
% bis zum ersten Abschnitt wird Kapitel ohne Nummer angezeigt
\renewcommand*{\chaptermark}[1]{\markboth{\chaptermarkformat#1}{}}
 
\makeatletter
% damit die letzte rechte Marke auf einer Seite genommen wird
\providecommand*{\rightbotmark}{\expandafter\@rightmark\botmark\@empty\@empty}
% damit \leftmark voll expandierbar ist (wird für \ifstrg benötigt)
\renewcommand*{\@seccntformat}[1]{%
  \othersectionlevelsformat{#1}{}{\csname the#1\endcsname}%
}
\makeatother 
   
\clearpairofpagestyles
\cfoot*{\pagemark}
\ihead{\ifstr{\rightbotmark}{\leftmark}{}{\rightbotmark}}
\ohead{\leftmark}

% Listings
\usepackage{listings}
\usepackage{lstautogobble}
\usepackage{bera}
\usepackage{xcolor}


\colorlet{punct}{red!60!black}			% JSON highlighting: http://tex.stackexchange.com/a/83100
\definecolor{background}{HTML}{EEEEEE}
\definecolor{delim}{RGB}{20,105,176}
\colorlet{numb}{magenta!60!black}
\lstdefinelanguage{json}{
    basicstyle=\normalfont\ttfamily,
    numbers=left,
    numberstyle=\scriptsize,
    stepnumber=1,
    numbersep=8pt,
    showstringspaces=false,
    breaklines=true,
    frame=lines,
    backgroundcolor=\color{background},
    literate=
     *{0}{{{\color{numb}0}}}{1}
      {1}{{{\color{numb}1}}}{1}
      {2}{{{\color{numb}2}}}{1}
      {3}{{{\color{numb}3}}}{1}
      {4}{{{\color{numb}4}}}{1}
      {5}{{{\color{numb}5}}}{1}
      {6}{{{\color{numb}6}}}{1}
      {7}{{{\color{numb}7}}}{1}
      {8}{{{\color{numb}8}}}{1}
      {9}{{{\color{numb}9}}}{1}
      {:}{{{\color{punct}{:}}}}{1}
      {,}{{{\color{punct}{,}}}}{1}
      {\{}{{{\color{delim}{\{}}}}{1}
      {\}}{{{\color{delim}{\}}}}}{1}
      {[}{{{\color{delim}{[}}}}{1}
      {]}{{{\color{delim}{]}}}}{1}
      {Ö}{{\"O}}1
      {Ä}{{\"A}}1
      {Ü}{{\"U}}1
      {ß}{{\ss}}1
      {ü}{{\"u}}1
      {ä}{{\"a}}1
      {ö}{{\"o}}1
      {~}{{\textasciitilde}}1,
}

\lstset{
	tabsize=2,
	basicstyle=\ttfamily,
	autogobble}
