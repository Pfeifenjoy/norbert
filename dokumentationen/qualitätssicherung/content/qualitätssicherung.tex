%!TEX root = ../documentation.tex

\chapter{Qualitätssicherung}
Das folgende Kapitel beschreibt unterschiedliche Prozesse, die dazu beitragen die Softwarequalität zu steigern. 
Hierzu sollen zunächst die Qualitätsziele definiert werden und anschließend Maßnahmen zur Umsetzung dieser beschrieben werden.

\section{Qualitätsziele}
Neben den vertraglich vereinbarten Qualitätsziele, die in dem Dokument \enquote{Pflichtenheft-norbert-v2} unter Kapitel 9 \enquote{Qualitätsziele} zu finden sind, wird bei der Qualität auf übersichtlichen und verständlichen Quellcode, Portabilität und Vollständigkeit geachtet.

\subsection{Vollständigkeit}
Basierend auf dem Pflichtenheft sollen alle Funktionen implementiert sein, die als \enquote{muss-Kriterien} (Priorität 1) gesetzt sind.

\subsection{Portabilität}
Durch die web-orientierte Anwendung gewährleistet Norbert Zugriff von verschiedenen Endgeräten im Browser. 

\subsection{Übersichtlicher und verständlicher Quellcode}
Der Quellcode ist übersichtlich strukturiert und mit Kommentaren versehen. Weitere Einzelheiten zu den Code-Conventions sind dem Dokument \enquote{Softwareentwurf-norbert-v2} Kapitel 7 \enquote{Code Conventions} zu entnehmen.

\newpage
\section{Maßnahmen zur Qualitätssicherung}
\begin{tabularx}{\textwidth}{|l|X|X|}
    \toprule
    \textbf{ID-Kürzel} & \textbf{Beschreibung} & \textbf{Begründung}\\
    \midrule
    \endhead
    \hline
    \caption{Qualitätssicherung}    
    \endfoot    
    Q-10 & Code-Reviews & Code-Reviews werden genutzt, um den erstellten Code durch ein anderes Teammitglied überprüfen zu lassen. Die Fehlerrate wird 		dadurch gesenkt und es kann die Vorgehensweise diskutiert werden. Mögliche Änderungen können dadurch schnell getätigt werden und haben nur 		geringe Auswirkung auf die Anwendung.  \\ \hline
    Q-20 & Pair-Programming & Beim Pair-Programming ist der Vorteil, dass sich schnell gegenseitig abgestimmt und gemeinsam ein Lösungsweg  		gefunden wird. Man beeinflusst sich gegenseitig und findet daher noch bessere Lösungswege.\\ \hline
    Q-30 & Team-Meetings & Team-Meetings dienen zur Abstimmung innerhalb des Entwicklungsteam. Ein solches Meeting vermeidet z.B. Duplikate im 		Code. Außerdem wird in den Team-Meetings gezeigt, wie ein Interface einer anderen Komponente genutzt wird. Daher steigern Team-Meetings die 	Effizienz, wenn bereits bekannte Komponenten verwendet werden können.\\  
\end{tabularx}

\section{Relevante Dokumente zur Softwarequalität}


Mit Hilfe der folgenden Dokumenten wurde die Softwarequalität spezifiziert.
\begin{enumerate}
\item Pflichtenheft
\item Softwareentwurf


\end{enumerate}






