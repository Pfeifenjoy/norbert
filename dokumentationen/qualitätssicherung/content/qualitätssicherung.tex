%!TEX root = ../documentation.tex

\chapter{Qualitätssicherung}
Das folgende Kapitel beschreibt unterschiedliche Prozesse, die dazu beitragen die Code Qualität zu steigern. 





\begin{tabularx}{\textwidth}{|l|X|X|}
    \toprule
    \textbf{ID-Kürzel} & \textbf{Beschreibung} & \textbf{Begründung}\\
    \midrule
    \endhead
    \hline
    \caption{Code Conventions}
    
    \endfoot
    
    Q-10 & Code-Reviews & Wird genutzt, um den erstellten Code durch jemand anderes überprüfen zu lassen. \\ \hline
    Q-20 & Pair-Programming &  fsds\\ \hline
    C-30 & Team-Meetings & Dient zur Abstimmung innerhalb des Entwicklungsteam. Ein solches Meeting vermeidet z.B. Duplikate im Code. Außerdem wird in den Team-Meetings gezeigt, wie ein Interface einer anderen Komponente genutzt wird.\\
	
 
  
\end{tabularx}