%!TEX root = ../documentation.tex

\chapter{Testspezifikation}
Das folgende Kapitel beschreibt verschiedene Testszenarien.



\begin{tabularx}{\textwidth}{|l|X|X|}
    \toprule
    \textbf{ID-Kürzel} & \textbf{Beschreibung} & \textbf{Erwartetes Ergebnis} \\
    \midrule
    \endhead
    \hline
    \caption{Testspezifikationen}
    
    \endfoot
    \multicolumn{3}{|l|}{Authentifizierung}\\ \hline
    T-A-10 & Benutzer registrieren & Ein Benutzer ist registriert, wenn der eingegebene Benutzername eindeutig ist und die beiden eingegebenen Passwörter übereinstimmen und die vorgegebene Passwortlänge (siehe A-50) einhalten wird. Falls nicht, wird eine Fehlermeldung am Bildschirm ausgegeben. \\ \hline
    T-A-20 & Anmelden &  Der Nutzer gibt einen gültigen Benutzernamen, sowie das dazugehörige Passwort ein und wird auf die Hauptseite weitergeleitet. \newline Falls eine fehlerhafte Eingabe (Benutzername nicht bekannt, falsches Passwort) getätigt wurde, wird ein Fehler am Bildschirm ausgebeben.\\ \hline
    T-A-30 & Benutzername eindeutig & Ein Benutzername darf nur einmal existieren. Falls bereits ein Benutzername existiert, wird eine Fehlermeldung ausgeben.\\ \hline
    T-A-40 & Passwort Länge & Das Passwort muss mindestens eine Länge von 10 Zeichen haben. Andernsfalls ist eine Registrierung nicht möglich.  \\\hline
    T-A-50 & Logout & Nach dem Logout wird dem Benutzer die Loginseite angezeigt.\\ \hline
    \multicolumn{3}{|l|}{Einträge}\\ \hline
    T-E-10 & Eintrag anlegen &  Nachdem ein Eintrag angelegt wurde, wird dieser im Newsfeed angezeigt. \\\hline
   	T-E-20 & Eintrag bearbeiten & Falls ein Eintrag bearbeitet wurde, werden die Änderungen des Eintrags im Newsfeed angezeigt. \\\hline
    T-E-30 & Eintrag löschen &  Wird ein Eintrag gelöscht, verschwindet er aus dem Newsfeed.\\\hline
    T-E-40 & Text & Wird einem Eintrag ein beschreibender Text hinzugefügt, so wird die Beschreibung im Eintrag des Newsfeeds angezeigt. \\ \hline
    T-E-50 & Checkliste/Task & Wird eine Checkliste einem Eintrag hinzugefügt, so wird diese in dem Eintrag im Newsfeed angezeigt. \\ \hline
    T-E-60 & File & Wird an einen Eintrag eine Datei angehängt, dann wird im Eintrag ein Symbol angezeigt, das signalisiert, dass ein Anhang existiert. \\ \hline
 	T-E-70 & Erinnerung & Wird an einem Eintrag eine Erinnerung eingetragen, so wird der Eintrag im Newsfeed mit einem Glocken-Symbol angezeigt.\\ \hline
    \multicolumn{3}{|l|}{Vorschläge}\\ \hline
    T-V-10 & Vorschlag übernehemen  & Wird ein Vorschlag akzeptiert, so wird dieser im eigenen Newsfeed angezeigt. \\ \hline
    T-V-20 & Vorschlag ablehnen & Wird ein Vorschlag abgelehnt, dann wird dieser Vorschlag nicht mehr im Newsfeed angezeigt. \\ \hline
    \multicolumn{3}{|l|}{Informationen}\\\hline
    
    T-I-10 & Informationen können ausgeblendet werden & \\ \hline
	\multicolumn{3}{|l|}{Suche}\\ \hline
	T-S-10 & Schlagwortsuche  & Wird nach einem Schlagwort, dass in die Suchleiste eingegeben wird, gesucht, so werden die passenden Einträge im Newsfeed angezeigt.\\  

    
  
	
 
  
\end{tabularx}



