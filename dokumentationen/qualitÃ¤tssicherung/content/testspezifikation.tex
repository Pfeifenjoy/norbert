%!TEX root = ../documentation.tex

\chapter{Testspezifikation}
Das folgende Kapitel beschreibt verschiedene Testszenarien.



\begin{tabularx}{\textwidth}{|l|X|X|X|}
    \toprule
    \textbf{ID-Kürzel} & \textbf{Beschreibung} & \textbf{Erwartetes Ergebnis}& \textbf{Kommentar} \\
    \midrule
    \endhead
    \hline
    \caption{Testspezifikationen}
    
    \endfoot
    \multicolumn{4}{|l|}{User Daten}\\ \hline
    T-U10 & Benutzername eindeutig & Hallo & FDJKS \\ \hline
    T-U20 & Passwort Länge & & \\\hline
    \multicolumn{4}{|l|}{Einträge}\\ \hline
    T-E10 & Eintrag anlegen & & \\\hline
   	T-E20 & Eintrag bearbeiten & & \\\hline
    T-E30 & Eintrag löschen & & \\\hline
    \multicolumn{4}{|l|}{Dateiupload} \\\hline
    T-D10 & Datei in Dropbox hochladen & & \\ \hline
    \multicolumn{4}{|l|}{ICal Datei }\\ \hline
    T-I10 & .ical Einträge als Notification anzeigen & & \\

    
  
	
 
  
\end{tabularx}



