%!TEX root = ../documentation.tex

\chapter{Testspezifikation}
Das folgende Kapitel beschreibt verschiedene Testszenarien.

\section{T-10 - Nutzer erstellen und Eintrag anlegen}
\subsection{Ziel}
Die Test deckt die Hauptfunktionalität von Norbert - Your StudyBuddy ab. Zuerst wird ein User angelegt. Nachdem sich der Nutzer eingelogt hat, erstellt er einen Test-Eintrag.

\subsection{Voraussetzung}
Norbert läuft auf einem Server. 

\subsection{Test Durchführung}
\begin{enumerate}
\item Öffenen sie eine Webbrowser ihrer Wahl.
\item Geben Sie in die Adressleiste folgenden Link ein: LOCALHOST:3000. \\Es öffnet sich die Startseite von Norbert.
\item Wechseln Sie zum Regitrierungs Dialog, indem sie auf \enquote{registrieren} klicken. Es öffnent sich ein Formular \enquote{Registrieren}.
\item Geben Sie in dem Feld \enquote{User} \textbf{user01} ein.
\item Geben Sie in dem Feld \enquote{Passwort} \textbf{norbert123} ein.
\item Geben Sie in dem Feld \enquote{Passwort wiederholen} \textbf{norbert123} ein.
\item Klicken sie auf den Button registrieren. Sie haben nun einen Account erfolgreich angelegt.
\item Legen sie nun einen Eintrag an.
\begin{enumerate}
\item Klicken sie auf das \enquote{+} Zeichen im rechten unteren Bereich des Bildschirms..
\item Es öffnet sich ein Popup.
\item Eingaben
\item Eingaben
\end{enumerate}
\item Klicken sie auf speichern. Der Eintrag wird angezeigt.
\item Ändern sie den Titel.
\begin{enumerate}
\item Eintrag bearbeiten button.Es öffnet sich ein Pop-up
\item Bearbeiten sie den Titel zu: 
\item Klicken sie speichern, um die Änderungen zu speichern.
\end{enumerate}
\item Der Eintrag hat nun den Titel ???.
\item Um sich aus zu loggen, klicken sie in der rechten oberen Ecke auf \enquote{Logout}.
\end{enumerate}

