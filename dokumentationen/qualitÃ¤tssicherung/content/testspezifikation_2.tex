%!TEX root = ../documentation.te

\chapter{Testspezifikation - Nutzer erstellen und Eintrag anlegen}
\section{Ziel}
Die Test deckt die Hauptfunktionalität von Norbert - Your StudyBuddy ab. Zuerst wird ein User angelegt. Nachdem sich der Nutzer angemeldet hat, erstellt er einen Eintrag, bearbeitet den Eintrag und meldet sich anschließend wieder ab.


\section{Voraussetzung}
Norbert läuft auf einem Server. 

\section{Test Durchführung}
\begin{enumerate}
\item Öffenen sie eine Webbrowser ihrer Wahl.
\item Geben Sie in die Adressleiste folgenden Link ein: http:localhost:3000. \\Es öffnet sich die Startseite von Norbert.
\item Wechseln Sie zum Regitrierungsformular, indem sie auf \enquote{register} klicken. Es öffnet sich ein Formular \enquote{Registrieren}.
\item Geben Sie in dem Feld \enquote{Username} \textbf{user01} ein.
\item Geben Sie in dem Feld \enquote{Password} \textbf{norbert123} ein.
\item Wiederholen sie die Passworteingabe im Feld \enquote{RePassword}: \textbf{norbert123} .
\item Klicken sie auf den Button \enquote{Register}. Sie haben nun einen Account erfolgreich angelegt und werden automatisch zur Hauptseite weitergeleitet. 
\item Legen sie nun einen Eintrag an.
\begin{enumerate}
\item Klicken sie auf das \enquote{+} Zeichen, um einen neuen Eintrag zu Erstellen.
\item Es wird ein neuer Eintrag mit dem Titel \enquote{neuer Eintrag} im Newsfeed angelegt.
\item Klicken sie auf die rechte obere Ecke bei dem unter Punkt b) erstellen Eintrag.\\ Es öffnet sich ein Popup.
\item Ersetzen sie \enquote{Neuer Eintrag} in der Eingabeleiste durch: \textbf{ Mein erster Eintrag}.
\item Schließen Sie das Pop-up durch drücken der ESC-Taste.\\ Der von Ihnen erstelle Eintrag wird nun im Newsfeed mit dem Titel \enquote{Mein erster Eintrag} angezeigt.
\end{enumerate}
\item Fügen sie dem Eintrag eine Beschreibung hinzu. Öffnen Sie den Eintrag durch Drücken des Stift-Symbol im rechten oberen Eck des Eintrags. Es öffnet sich wieder das Pop-up, damit der Eintrag bearbeitet werden kann.
\begin{enumerate}
\item Wählen Sie in dem Dropdown Menü den Menüpunkt \enquote{Beschreibung} aus.
\item Ein Textfeld wird dem Eintrag hinzugefügt.
\item Schreiben Sie nun in das Textfeld: \textbf{Das ist eine Beschreibung!}. 
\item Schließen Sie das Pop-up durch Drücken der ESC-Taste.
\end{enumerate}
\item Im Newsfeed ist nun der Eintrag mit dem Titel \enquote{Mein erster Eintrag} und der Beschreibung \enquote{Das ist eine Beschreibung!} zu sehen.
\item Melden Sie sich nun von Norbert ab.
\begin{enumerate}
\item Klicken Sie im rechten oberen Eck der Webseite, in der Menüleiste, auf \enquote{Verlassen}.
\item Sie sind nun abgemeldet und sehen nun wieder die Login-Seite. \\Dort können Sie sich mit ihrem angelegten Benutzernamen \textbf{user01} und dem vergebenen Passwort \textbf{norbert123} wieder anmelden.
\end{enumerate}
\end{enumerate}

